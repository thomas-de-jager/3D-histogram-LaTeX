\documentclass[class=article]{standalone}
\usepackage[utf8]{inputenc}
\usepackage{tikz}
\usetikzlibrary{calc}
\usepackage{pgfplots}
\usepackage{pgfplotstable}
\pgfplotsset{compat=1.8}


% Used to make 3D-histogram
% from https://tex.stackexchange.com/a/102770/121799
\def\pgfplotsinvokeiflessthan#1#2#3#4{%
    \pgfkeysvalueof{/pgfplots/iflessthan/.@cmd}{#1}{#2}{#3}{#4}\pgfeov
}%
\def\pgfplotsmulticmpthree#1#2#3#4#5#6\do#7#8{%
    \pgfplotsset{float <}%
    \pgfplotsinvokeiflessthan{#1}{#4}{%
        % first key <:
        #7%
    }{%
        \pgfplotsinvokeiflessthan{#4}{#1}{%
            % first key >:
            #8%
        }{%
            % first key ==:
            \pgfplotsset{float <}%
            \pgfplotsinvokeiflessthan{#2}{#5}{%
                % second key <
                #7%
            }{%
                \pgfplotsinvokeiflessthan{#5}{#2}{%
                    % second key >
                    #8%
                }{%
                    % second key ==
                    \pgfplotsset{float <}%
                    \pgfplotsinvokeiflessthan{#3}{#6}{%
                        % third key <
                        #7%
                    }{%
                        % third key >=
                        #8%
                    }%
                }%
            }%
        }%
    }%
}%



\begin{document}
    \ifdefined\gconv
    \else
        \pgfmathsetmacro{\gconv}{0.1}
    \fi
    
    \newcommand\maxZValue{10} % manually change this value to max z value in dataset!
    
    \pgfplotstableread[col sep=comma,header=true]{%
        Data/paperYearData.csv%
    }{\datatable}
    
        \pgfplotstablesort[create on use/sortkey/.style={
            create col/assign/.code={%
                \edef\entry{{\thisrow{x}}{\thisrow{y}}{\thisrow{myvalue}}}%
                \pgfkeyslet{/pgfplots/table/create col/next content}\entry
            }
        },
        sort key=sortkey,
        sort cmp={%
            iflessthan/.code args={#1#2#3#4}{%
                \edef\temp{#1#2}%
                \expandafter\pgfplotsmulticmpthree\temp\do{#3}{#4}%
            },
        },
        sort,
        columns/Mtx/.style={string type},
        columns/Kind/.style={string type},]\resulttable{\datatable} {
    % \small

    \begin{tikzpicture}%[x={(0.866cm,-0.5cm)},y={(0.866cm,0.5cm)},z={(0cm,1 cm)}]
        \pgfplotsset{set layers}
        \begin{axis}[% from section 4.6.4 of the pgfplotsmanual
            view={165}{50},
            width=440pt, % was: 440pt
            height=380pt, % was: 380pt
            z buffer=none,
            title={Amount of selected papers per category with respect to their publication year},
            xmin=0,xmax=6,
            ymin=1999,ymax=2020,
            ylabel style={align=center},
            zmin=0,zmax=\maxZValue*2,
            enlargelimits=upper,
            ztick={0,3*2,6*2,9*2},
            zticklabels={0,3,6,9}, % here one has to "cheat"
            % zticklabels={0,\pgfmathparse{1/3*\maxZValue}\pgfmathresult,\pgfmathparse{2/3*\maxZValue}\pgfmathresult,\pgfmathparse{3/3*\maxZValue}\pgfmathresult}, % here one has to "cheat"
            % meaning that one has to put labels which are the actual value 
            % divided by 2. This is because the bars will be centered at these
            % values
            xtick=data,
            xticklabels={Industrial,Tendon,Origami,Metamaterial,Linkage,Compliant},
            extra tick style={grid=major},
            % ytick=data,
            ytick={2000,2005,2010,2015,2020},
            yticklabels={2000,2005,2010,2015,2020},
            grid=major, %changes lines on axes (was: minor)
            major grid style={gray!60},
            xlabel={Category},
            ylabel={Year~of\\publication},
            zlabel={Amount~of~papers},
            z tick label style={
                anchor=east}, % used to align label with axis line
            y tick label style={
                anchor=east},
            minor x tick num=0,
            minor y tick num=4,
            minor z tick num=3,
            point meta=explicit,
            % define the custom colormap. from: https://tex.stackexchange.com/questions/370960/custom-colorbar-not-showing-colors-correctly/370962
            colormap={my colormap}{
                rgb=(1, 0, 0),
                rgb=(1, 1, 0),
                rgb=(0, 0, 1),
                rgb=(0.59, 0.29, 0.0),
                rgb=(0, 1, 0),
                rgb=(0, 0.5, 0.5),
            },
            % colormap name=viridis,
            scatter/use mapped color={
                % fill=blue!70!black,fill opacity=.1
                draw=black,fill=mapped color!80,fill opacity=.75
                % \definecolorseries{test}{rgb}{step}[rgb]{.95,.85,.55}{.17,.47,.37}
                % \resetcolorseries{test}
            },
            execute at begin plot={}
            ]
            \path let \p1=($(axis cs:0,0,1)-(axis cs:0,0,0)$) in 
            \pgfextra{\pgfmathsetmacro{\conv}{2*\y1}
            \ifx\gconv\conv
            \else
                \xdef\gconv{\conv}
                \typeout{Please\space recompile\space the\space file!}
            \fi     
            };
            \path let \p1=($(axis cs:1,0,0)-(axis cs:0,0,0)$) in 
            \pgfextra{\pgfmathsetmacro{\convx}{veclen(\x1,\y1)}
            \typeout{One\space unit\space in\space x\space direction\space is\space\convx pt}
            };                  
            \path let \p1=($(axis cs:0,1,0)-(axis cs:0,0,0)$) in 
            \pgfextra{\pgfmathsetmacro{\convy}{veclen(\x1,\y1)}
            \typeout{One\space unit\space in\space y\space direction\space is\space\convy pt}
            };                  
            \addplot3 [visualization depends on={\gconv*z \as \myz}, % you may have to recompile to get the prefactor right
            scatter/@pre marker code/.append style={/pgfplots/cube/size z=\myz},%
            scatter/@pre marker code/.append style={/pgfplots/cube/size x=12pt},% used to be: x=11.66135pt
            scatter/@pre marker code/.append style={/pgfplots/cube/size y=6pt},% used to be: y=9.10493pt
            scatter,only marks,
            mark=cube*,mark size=5,opacity=1]
            table[x expr={\thisrow{x}},y expr={\thisrow{y}},z
            expr={1*\thisrow{myvalue}},
            meta expr={-1*\thisrow{x}}
            ] \resulttable;
        \end{axis}
        \makeatletter
            \immediate\write\@mainaux{\xdef\string\gconv{\gconv}\relax}
        \makeatother
    \end{tikzpicture}
    }
\end{document}
